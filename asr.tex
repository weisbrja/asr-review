\documentclass[11pt]{article}

% Change "review" to "final" to generate the final (sometimes called camera-ready) version.
% Change to "preprint" to generate a non-anonymous version with page numbers.
\usepackage[final]{acl}

% Standard package includes
\usepackage{times}
\usepackage{latexsym}

\usepackage[english]{babel}
\usepackage{csquotes}

% For proper rendering and hyphenation of words containing Latin characters (including in bib files)
\usepackage[T1]{fontenc}
% For Vietnamese characters
% \usepackage[T5]{fontenc}
% See https://www.latex-project.org/help/documentation/encguide.pdf for other character sets

% This assumes your files are encoded as UTF8
\usepackage[utf8]{inputenc}

% This is not strictly necessary, and may be commented out,
% but it will improve the layout of the manuscript,
% and will typically save some space.
\usepackage{microtype}

% This is also not strictly necessary, and may be commented out.
% However, it will improve the aesthetics of text in
% the typewriter font.
\usepackage{inconsolata}

%Including images in your LaTeX document requires adding
%additional package(s)
\usepackage{graphicx}

\title{A Review of Automatic Speech Recognition}

\author{Jannis Weisbrodt \\
  KIT \\
  \texttt{jannis.weisbrodt@student.kit.edu}
  }

\begin{document}
\maketitle
\begin{abstract}
Abstract goes here.
\end{abstract}

\section{Introduction}

Automatic Speech Recognition (ASR) is the task of transcribing spoken language into writing -- one of the earliest goals of computer language processing \cite{jurafskySpeechLanguageProcessing2025}.
Today, automatic transcription is used ubiquitously not only in fields like law, where dictation is essential, but also in voice-controlled assistants as well as automatic captioning for movies, videos, conferences or voice messages.
Advances in ASR not only enhance accessibility for individuals with typing or hearing impairments, but also generally provide a more natural interface for interacting with modern machines.
Despite the progress, accurate automatic transcription of speech by any speaker in any environment still remains a challenging problem that is far from solved.
The ASR task is often complicated by background noise as well as varying speaker counts, accents and styles, further compounded by the large number of spoken languages and made even more complex by code-switching.
These challenges are further exacerbated in low-resource scenarios, where annotated speech data are limited---a situation that affects many languages, especially those spoken in less wealthier nations.
Additionally, large vocabularies pose significant difficulties for ASR systems, as some words are almost guaranteed to be underrepresented in the training data, increasing the complexity of recognition and modeling.
Spoken language also frequently contains disfluencies, such as fragmented, repeated or truncated words (e.g. \enquote{main- mainly}), as well as filled pauses (e.g. \enquote{uhm}, \enquote{uh}).

In this review, we will describe and compare state-of-the-art ASR systems, as well as challenges remaining to be solved.

\bibliography{zotero}

\end{document}
